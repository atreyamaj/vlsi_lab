% !TeX spellcheck = en_GB
\documentclass[12pt,a4paper]{article}
\usepackage[parfill]{parskip}
\usepackage{graphicx}
\usepackage{tabu}
\usepackage{amsmath}
\usepackage{float}
\usepackage{subcaption}
\newcommand{\ud}{\,\mathrm{d}}
\begin{document}
\thispagestyle{empty}
\begin{center}
\large{VLSI DESGIN LABORATORY\\
}
\large{LAB REPORT\\
\vspace{40pt}
CMOS INVERTER}
\end{center}
\vspace{80pt}

\begin{center}
\textbf{Submitted to}\\Mr. Kirankumar Lad
\vspace{80pt}
\textbf{Group Members} :\\
Anirudh B H (16EC105)\\
Manan Sharma (16EC118)\\


\end{center}

\newpage
\thispagestyle{empty}
\setcounter{page}{0}
\tableofcontents
\clearpage

\newpage
\begin{center}
\huge Characterization of CMOS Inverter
\end{center}

\section{Objectives}
Study the:
\begin{enumerate}
    \item Transfer function
    \item Noise Margin
    \item Risetime and Fall time
    \item Propagation delay
    \item Power and Energy consumed
\end{enumerate}
with variation in $R_L$ and W of the Load Resistor and Pull-Down Transistor. Along with this also calculate power and energy consumed for non ideal step input for Resistive Load Inverter.

\section{Introduction}
Inverter is one that inverts the signal supplied at its input. If input is made high or logic level is 1, then the output has a logic level 0 and vice-versa. A resistive load inverter is characterised by an resistive load between the pull down transistor and the voltage source. The output is taken at the junction of the load resistance and the pull-down transistor. The pull-up circuit is constituted by the resistor and the pull-down resistor by the NMOS. When the input is low, the NMOS is open circuited and the output capacitance is charged to $V_{DD}$ through $R_L$. 

\section{Netlist}

* Voltage sources \\
V1 Vdd 0 5 \\
V2 IN 0 5 \\

* Transistor definition \\
M1 OUT IN Vdd Vdd CMOSP W=4u L=2u \\
M2 OUT IN 0 0 CMOSN W=2u L=2u \\
C OUT 0 10f \\

The above statements represent the SPICE netlist for the CMOS inverter where IN represents the input and OUT represents output.

\section{Schematic}

\begin{figure}[H]
\begin{center}
\includegraphics[scale=0.2]{placeholder.jpg}
\caption{Schematic of CMOS Inverter}
\end{center}
\end{figure}


\section{Analysis}

Transfer functions were plotted for varying values of L and W of the driver and varying the $R_L$ value of the load resistor and the plots obtained are as follows:\\

The curves moving to the right implies the switching threshold is changing and moving towards the right (increasing) and vice-versa.\\

\begin{figure}[H]
\begin{center}
\includegraphics[scale=0.20]{placeholder.jpg}
\caption{Transfer function when width of PMOS is varied}
\end{center}
\end{figure}


\begin{figure}[H]
\begin{center}
\includegraphics[scale=0.20]{placeholder.jpg}
\caption{Transfer function when length of PMOS is varied}
\end{center}
\end{figure}

As the length of the PMOS is increased, width remaining constant, the ratio Kd/Kl (driver/load) increases and hence the curve shifts to the left.

\begin{figure}[H]
\begin{center}
\includegraphics[scale=0.20]{placeholder.jpg}
\caption{Transfer function when width of NMOS is varied}
\end{center}
\end{figure}

As width of the NMOS is increased, length remaining constant, the ratio Kd/Kl (driver/load) increases and due to this the transfer function curves shift to the left in this setting. 

\begin{figure}[H]
\begin{center}
\includegraphics[scale=0.20]{placeholder.jpg}
\caption{Transfer function when length of NMOS is varied}
\end{center}
\end{figure}

As the length of the NMOS is increased, width remaining constant, the ratio Kd/Kl (driver/load) decreases and hence the curve shifts to the right.\\


Transient characteristics for varying widths and lengths of PMOS and NMOS were plotted and the plots as as below:

\begin{figure}[H]
\begin{center}
\includegraphics[scale=0.20]{placeholder.jpg}
\caption{Transient behaviour(zoomed in) when length of PMOS is varied}
\end{center}
\end{figure}

When the length of PMOS is increased, the current through it reduces which in turn increases the resistance offered by it during charging resulting in a large time constant. Hence the curves are as seen in the plot.

\begin{figure}[H]
\begin{center}
\includegraphics[scale=0.20]{placeholder.jpg}
\caption{Transient behaviour when width of PMOS is varied}
\end{center}
\end{figure}

When the width of PMOS is increased, the current through it increases which in turn decreases the resistance offered by it during charging resulting in a small time constant. Hence the curves are as seen in the plot.

\begin{figure}[H]
\begin{center}
\includegraphics[scale=0.20]{placeholder.jpg}
\caption{Transient behaviour when length of NMOS is varied}
\end{center}
\end{figure}

When the length of NMOS is increased, the current through it decreases which in turn increases the resistance offered by it during discharging resulting in a large time constant. Hence the curves are as seen in the plot. The last curve is the one with largest time constant, takes a longer time to discharge.


\begin{figure}[H]
\begin{center}
\includegraphics[scale=0.20]{placeholder.jpg}
\caption{Transient behaviour when width of NMOS is varied}
\end{center}
\end{figure}

\section{Observations and Results}

The parameters such as Noise Margin, Rise time, Fall time, Propagation delay, Power dissipation are analyzed from the wave-forms and tabulated as follows:\\

All the timings parameters are in the order of 10 to the power of -11 and power dissipation in microwatt($\mu$ W).\\

Voh for CMOS inverter is nearly 5V and Vol for CMOS inverter is nearly 0V as CMOS inverter achieves a perfect zero and a perfect 1.\\

When width of driver transistor (NMOS) is varied:
\begin{table}[H]
\centering
\begin{tabular}{||c c c c c c c||} 
 \hline
 Parameter & 0.25u & 0.5u & 2u & 5u & 7.5u & 10u\\ [0.5ex] 
 \hline\hline
 NML & 3.60 & 3.16 & 1.92 & 0.94 & 0.644 & 0.524\\ 
 NMH & 0.75 & 1.03 & 2.02 & 2.82 & 3.15 & 3.36\\
 trise & 3.59 & 3.69 & 3.69 & 3.77 & 3.71 & 4.14\\
 tfall & 19.9 & 10.3 & 3.46 & 2.48 & 2.5 & 2.27\\
 tp & 9.08 & 5.275 & 2.691 & 2.117 & 2.073 & 1.954\\
 Power & 8.25 & 8.935 & 10.845 & 13.425 & 15.63 & 17.05 \\
 [0.5ex] 
 \hline
\end{tabular}
\caption{Effect of varying width of NMOS}
\label{table:1}
\end{table}

When length of driver transistor (NMOS) is varied:
\begin{table}[H]
\centering
\begin{tabular}{||c c c c c c||} 
 \hline
 Parameter & 0.25u & 0.5u & 1u & 2u & 3u \\ [0.5ex] 
 \hline\hline
 NML & 1.922 & 2.38 & 2.75 & 3.11 & 3.3\\ 
 NMH & 2.014 & 1.68 & 1.28 & 0.91 & 0.75\\
 trise & 3.69 & 3.57 & 3.24 & 2.71 & 2.86\\
 tfall & 3.46 & 3.99 & 6.29 & 12.39 & 20.19\\
 tp & 2.691 & 3.139 & 3.97 & 5.67 & 7.51\\
 Power & 10.845 & 9.85 & 10.8 & 13.03 & 15.6\\
 [0.5ex] 
 \hline
\end{tabular}
\caption{Effect of varying length of NMOS}
\label{table:1}
\end{table}

When width of load transistor (PMOS) is varied:
\begin{table}[H]
\centering
\begin{tabular}{||c c c c c c c||} 
 \hline
 Parameter & 0.25u & 0.5u & 2u & 5u & 7.5u & 10u\\ [0.5ex] 
 \hline\hline
 NML & 0.329 & 0.407 & 1.00 & 1.94 & 2.33 & 2.59\\ 
 NMH & 4.017 & 3.731 & 2.8 & 2.01 & 1.676 & 1.47\\
 trise & 36.64 & 18.7 & 5.97 & 3.53 & 3.28 & 3.16\\
 tfall & 3.26 & 3.3 & 3.33 & 3.44 & 3.44 & 3.63\\
 tp & 12.253 & 6.175 & 2.92 & 2.6 & 2.58 & 2.746\\
 Power & 6.3 & 6.45 & 8.3 & 10.7 & 12.25 & 13.95\\
 [0.5ex] 
 \hline
\end{tabular}
\caption{Effect of varying width of PMOS}
\label{table:1}
\end{table}

When length of load transistor (PMOS) is varied:
\begin{table}[H]
\centering
\begin{tabular}{||c c c c c c||} 
 \hline
 Parameter & 0.25u & 0.5u & 1u & 2u & 3u \\ [0.5ex] 
 \hline\hline
 NML & 1.921 & 1.29 & 0.686 & 0.446 & 0.385\\ 
 NMH & 2.02 & 2.66 & 3.15 & 3.56 & 3.745\\
 trise & 3.69 & 5.71 & 11.5 & 30.98 & 58.1\\
 tfall & 3.46 & 3.10 & 2.75 & 2.64 & 2.70\\
 tp & 2.687 & 3.45 & 5.07 & 8.09 & 11.57\\
 Power & 10.845 & 10.95 & 14.045 & 15 & 15.65\\
 [0.5ex] 
 \hline
\end{tabular}
\caption{Effect of varying length of PMOS}
\label{table:1}
\end{table}

The analysis was performed by for varying loads(capacitance) and non-ideal step input. Here by non ideal step input means duty cycle varied from the ideal. ON time is 25ns and OFF time is 75ns but total period remains 100ns.\\

\newpage
Non-ideal step input:
\begin{table}[H]
\centering
\begin{tabular}{||c c c c||}
 \hline
 Parameter & 5f & 10f & 15f\\ [0.5ex] 
 \hline\hline
 trise & 2.99 & 3.69 & 4.50\\ 
 tfall & 2.63 & 3.41 & 4.269\\
 tp & 2.184 & 2.691 & 3.066\\
 Power & 8.35 & 10.84 & 12.985\\
 [0.5ex] 
 \hline
\end{tabular}
\caption{Non ideal step input results}
\label{table:1}
\end{table}

Ideal step input:
\begin{table}[H]
\centering
\begin{tabular}{||c c c c||}
 \hline
 Parameter & 5f & 10f & 15f\\ [0.5ex] 
 \hline\hline
 trise & 2.99 & 3.69 & 4.50\\ 
 tfall & 2.63 & 3.41 & 4.26\\
 tp & 2.175 & 2.689 & 3.06\\
 Power & 4.035 & 4.07 & 4.06\\
 [0.5ex] 
 \hline
\end{tabular}
\caption{Ideal step input results}
\label{table:1}
\end{table}

\section{Conclusion}
The experiment was performed and all parameters were extracted and analyzed. The values obtained agreed with the theory and hence simulations were verified. NGSPICE was the simulator used in this task.

\end{document}